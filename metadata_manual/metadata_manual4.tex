\chapter{作成したメタデータの提出と登録確認の方法}
 XML形式で記述されたIUGONETメタデータファイルの作成が出来たら、次にそれをメタデータ・データベースに登録
する為に、担当者またはメタデータ受付サーバーに提出します。この章では、作成したメタデータをどのように提出し、
確認するのかを説明します。

\section{電子メールによる提出}
 初期段階でのメタデータ登録は、電子メールにメタデータを添付することで行います。これは非常に簡単で、メタデータを
作成したディレクトリ構造を、そのままzip形式にアーカイブして、そのzipファイルをメタデータ登録用メールアドレスに
へ添付にて送信して下さい。($\_$at$\_$は@と読み替えて下さい。)。その際に、メールの
タイトルは、{\bf メタデータ登録\_YYYYMMDD\_機関名}とし、本文にはメタデータ作成者の名前と登録内容を簡潔に記載して下さい。
\par
 送られたメタデータは、IUGONETのサーバーにてチェックスクリプトにかけられます。エラーがあった場合、IUGONETからメタデータ
を送ったメールアドレス宛にエラー内容が記載されたログが送信されます。その内容を参考にエラーの修正を行った後、
再度アーカイブしてメタデータ登録用メールアドレスへ送りなおして下さい。

\section{Git\index{Git@Git}による提出}
大量のメタデータを取り扱うようになると、メタデータの効率的な管理がひとつの課題になります。IUGONETでは、メタデータ
受付サーバー上でのメタデータ登録管理に、分散型バージョン管理システムである{\bf Git\index{Git@Git}}を採用しています。
Git\index{Git@Git}によるメタデータ登録管理を始めるのは、各データファイルに紐付けされたGranule\index{Granule@Granule}メタデータを
作成する直前が良いでしょう。本節では、
Git\index{Git@Git}を使ったIUGONETメタデータ登録について簡単に紹介します。\par
 
\begin{screen}
\begin{verbatim}
01 $ git config apply.whitespace warn
02 $ git config --global user.name "Inuta Neko"
03 $ git config --global user.email "hogehage@mememe.huhuhu"
\end{verbatim}
\end{screen}

\begin{screen}
\begin{verbatim}
01 $ git add .
02 $ git add -u
03 $ git commit
04 $ git push
\end{verbatim}
\end{screen}

\begin{screen}
\begin{verbatim}
01 $ git pull
\end{verbatim}
\end{screen}

gitは非常に高機能なソフトウェアで、上記で説明した以外にも沢山の機能を持ちます。
より効率よく使いたい場合は、書籍やインターネット上のドキュメントを参考にして下さい。
